\documentclass{article}
\usepackage[utf8]{inputenc}
\usepackage{hyperref}
\usepackage[margin=1.5in]{geometry}
\usepackage{enumitem}
\usepackage[makeroom]{cancel}

\title{Advisor Guide: Answers to Common Questions}
\author{ThanhVu Nguyen\\\\\href{https://roars.dev}{Roars Lab}}
\date{}

\begin{document}

\maketitle


\tableofcontents

\newpage

Specific answers to questions about \href{https://roars.dev}{Roars Lab}. Adapted from \href{https://www.cs.columbia.edu/wp-content/uploads/2019/03/Get-Advisor.pdf}{this guide}. More general PhD admission advice can be found in my \href{https://code.roars.dev/phd-cs-us/}{PhD Admission Demystify} book.


\section{About the Advisor}

\begin{enumerate}[label=\arabic*.]
    \item \textbf{Does the professor have tenure yet?} [engagement level, PhD might get interrupted]  
    
    Yes, I am tenured. If you are wondering if I will be around for a while, the answer is yes. Northern Virginia is a great place to live and work, and Mason is growing fast and getting better every year. See more about \href{https://github.com/dynaroars/dynaroars.github.io/wiki/About-GMU}{CS@GMU}.  
    
    More importantly, having tenure allows me to take risks, pick problems I care about, and ignore the \textit{publish-or-perish} pressure that many non-tenured faculty face. That also means my students have to be independent and take initiative in their work because I won't be pushing them to do things.

    \item \textbf{What is the professor's formal training / background / PhD?} [helps contextualize problems/approaches]  
    
    I did my Ph.D. in CS in the University of New Mexico-Albuquerque and then a postdoc at the University of Maryland (in \href{https://plum-umd.github.io}{PLUM lab}). I started wanting to work on evolutionary computing, and then gradually changed to software engineering, programming languages, and formal methods. You can see my \href{https://raw.githubusercontent.com/dynaroars/latex-cv/main/cv.pdf}{CV} and other details from my \href{https://roars.dev}{lab} page and \href{https://tvn.roars.dev}{homepage}.

    \item \textbf{What have previous lab members done after getting their PhD?} [Gone to industry?/Post-doc?/Professor?]  
    
    So far I have only graduated one Ph.D. student: Guolong Zheng (2022), who is now a faculty at Minjiang University in China. Didier plans to graduate in May'26 and is applying for faculty positions.  
    
    I also graduated an MS student, who now works at Oracle, and an undergrad, who works at Jump Trading.
\end{enumerate}

\section{Group Composition \& Structure}

\begin{enumerate}[label=\arabic*.]
    \item \textbf{Do you work with undergraduate students?} [if you are an undergrad applicant]

    Yes, I work with many undergraduates over the years. Typically at any time I have 2--3 undergraduates working in the lab. 
    
    My undergrads are often supported either through research grants (e.g., NSF REU, university funding) or hourly pay. I treat my undergrads as PhD students, give them flexibility to find solutions (to projects that I give them), and push them (\emph{way}) beyond their comfort zone to realize their potential.
    
    As examples, Linhan started as an undergrad at UNL and continued as a PhD student GMU, KimHao published \emph{9 papers at top conferences and journals} as an undergrad at UNL, Stefania won the Outstanding Undergraduate Research Award at GMU, and Azan---a first year freshman when joining our lab---built \href{https://roars.dev/cspicks}{CSPicks} app within just a week of joining.

    See the \href{https://roars.dev#people}{Roars People section} for current undergrad members.

    \item \textbf{How many students are in the group?} [Number of undergrad/masters/phd/post doc]

    Check out the \href{https://roars.dev#people}{Roars People section}. Typically we have 5--6 PhD students and a couple of undergrads. We have yet to have a postdoc, but I would be interested in hosting one if there is a good match.

    \item \textbf{What is the lab structure?} [how collaborative/disjointed are lab members' projects?]  

    Each student (regardless of undergrad or PhD) has projects that they lead. However, everyone is encouraged to collaborate and help each other. In weekly lab meetings, you will hear about the progress of other projects and can contribute ideas and help with problems. We also have a lab server on Discord where we chat about research and other random things.  
    
    As examples, Didier works on Complexity analysis. KimHao on analyzing build systems, and also complexity analysis and invariant generation. Hai works on DNN verification. Linhan works on DNN testing. Hai, Linhan, and Nguyen often talk to each other as their projects involve neural networks. Long does not work with verification but contributes ideas and expertise as he knows ML systems very well. And everyone contributes to fun web apps (e.g., \href{https://roars.dev/cspicks}{CSPicks} by Azan).

    \item \textbf{Do students mostly work with senior students or directly with professor?}  

    They work directly with me, I encourage new lab members who are unsure about their projects to work with senior students with similar interests.
\end{enumerate}

\section{Advising Style \& Meetings}

\begin{enumerate}[label=\arabic*.]
    \item \textbf{Does the advisor consider themselves a ``hands-on'' or ``hands-off'' advisor?}  

    I am a \textit{hands-off} advisor. Students who need a lot of guidance and hand-holding \emph{will not} enjoy working with me.  
    I will help suggest ideas and directions, but I expect my students to be able to work independently and take ownership of their projects. My students, including undergrads, need to be self-motivated and take initiative in their work.  
    
    For new and junior students, I can provide some help (e.g., directions and writing). As students become more senior/mature, I would gradually transition to a hands-off advisor. However, if I see new members capable of working independently from day one, I will let them do so from day one -- I don't get in the way of capable students. For example, below is a direct quote of what I reply on Discord to a \textit{new undergrad} student (that I found highly capable just after a couple of weeks of working together and they already built a serious working prototype) when they questions about a technical detail:
    \begin{quote}
    \textit{``I am not sure what would be best. Do what you think is best and that would be best.''}
    \end{quote}

    However, although I am hands-off most of the time, I do step in when it's really needed. If a student is truly stuck, facing an urgent problem, or approaching a deadline, I instantly switch into a very hands-on (or protector) mode. This only happens a small fraction of the time—maybe 2\%—but in those moments I will work closely with you to get through the challenge.
    
    As mentioned, I highly encourage my students to talk to other lab members for help and guidance, e.g., more experienced students can help new students with ideas and research guidance. Our lab members are very close and collaborative: they challenge each other's ideas and help each other with problems.

    \item \textbf{How does the advisor give feedback on papers / what is their feedback style?}  
    
    For the students' first papers, I would ask for drafts and revise the draft iteratively with the students (e.g., through Overleaf). In some cases, I would rewrite most of the students' drafts, especially the Intro and Evaluation, for their first papers (because new students often do not know how to write these sections). This helps the students see how papers are written. Same thing with paper rebuttals, I will work with the students and revise the writing directly.  
    
    As the students have more experience, I will let go more and more. By the time the students can write and successfully publish the paper by themselves without much revision and editing from me, then I know they are ready to graduate.

    \item \textbf{Are there lab meetings? What are other meetings you will see your advisor in a group with other people?}  
    
    Yes, we meet weekly in the conference room on Thursday afternoon in the CS conference room. Everyone speaks, everyone contributes, no one is silent. Typically, the students talk about what they have been doing in the past week and what they will do next week. They also talk about issues they are facing and others contribute ideas to help solve the problems.  
    
    Occasionally after status updates, a student will present their work in depth. The meeting usually lasts about 1–-2 hours depending on how much we have to discuss. I am usually available after the meeting to chat with students individually if they have more questions.

    Sometimes we all come and work together in a large conference room (e.g., when we are close to a paper deadline), and then have lunch together.

    \item \textbf{What does a group/lab meeting look like?} [Or other relevant meetings]  
    
    We start talking about our status, e.g., each person talks about their work for about 2-–4 mins. This is inspired by the \href{https://mwhicks1.github.io/papers/score.pdf}{SCRUM method} used in the PLUM Lab at UMD.  
    
    Then we go in depth in some topic, e.g., a student might talk about some problem they are working on, show their computation on the board, or present results. Sometimes we read papers or look at some existing tools/techniques. We plan to use about an hour for this but usually it goes beyond that. I usually have to go pick up my kids at that time, but the students want to continue and they keep going.  

    My lab is highly collaborative and social among lab members. They often help each other with problems, which I highly encourage.  
    
    Sometimes we just relax and watch a movie (e.g., the PhD Movie).  
    
    Also, most of our communications happen on Discord server outside the meetings where we chat about research and other random things (e.g., plan for Thanksgiving party).

    \item \textbf{How often does the advisor meet with their students?} [1:1 or all together? Daily guidance by PI or post-doc?]  
    
    In addition to weekly lab meetings described above, students can talk to me individually about any issues they have. And during paper deadlines, we sometimes meet several times a day on Zoom.

    Many of my students work late at night and so do I (after kids have gone to bed!). So I've been making myself available to work with my students at night whenever they need, often 9:30 PM -- 11:30 (much later when I was younger).

    That said, I highly value independence, and so as long as it works for you and you're productive, it \textit{doesn't matter} to me how/when you work.

    \item \textbf{How often are students expected to be contactable by their advisor?}  
    
    Most of my communications with students happen over Discord. I sometimes contact students through email when it is something important, (e.g., the university asks me to get in touch with you about something). For such occasions, I expect you to respond to my emails quickly. However, in general, communications through Discord are more casual and you can respond when you see them --- though I find that all my students are quite responsive on Discord.
\end{enumerate}

\section{Research Areas \& Projects}

\begin{enumerate}[label=\arabic*.]
    \item \textbf{How directly applicable will your future technical skills be to the roles you want after graduating?} (If set on industry) 
    
    Skills you learn in my group (formal reasoning, verification, program analysis, being independent, writing well, etc) are real skills that are applicable to both academia, industry, and everywhere else.  
    
    Actually, this question is kind of weird (but it was on the \href{https://www.cs.columbia.edu/wp-content/uploads/2019/03/Get-Advisor.pdf}{original list}). We go for hard problems and they often coincide with real problems in the industry. We didn't set out to solve ``industry problems'' per se, but we often end up solving them anyway. For example, KimHao's work on build systems is directly applicable to large-scale software development at Facebook. Hai's work on DNN verification is applicable to many systems that use DNNs, which we believe will soon be everywhere (e.g., self-driving cars).

    \item \textbf{What research methods does the lab use?} (What \textit{types} of papers / contributions / conferences targeted)  
    
    Our work is often on developing new techniques/algorithms and building tools. We target top-tier conferences in the field (e.g., ICSE, FSE, ASE, ISSTA, OOPSLA, PLDI, CAV). For work we wish to extend, we also publish in journals such as IEEE TSE.  
    
    Ok, that was a boring neutral ``academic'' answer. The fact is that we don't just publish papers or produce boring incremental results, we also develop tools that actually work, win competitions, and receive significant awards. Our competitors are the best in the world and we enjoy \emph{beating them}. For example, \href{https://code.roars.dev/neuralsat}{NeuralSAT} is often considered one of the best at DNN verification competition and outperforms many other state-of-the-art tools.

    \item \textbf{What are some of the projects that you and your students are currently working on?}  
    
    We are working on various software analysis projects, including formalizing and proving mathematical theorems, neural network verification and inference, analysis on highly-configurable software, and program testing, analysis, and repair. 

    Ok, cross out \cancel{program repair} --- I moved on from that. Nowadays I am more interested in DNN verification and reasoning and also using LLM-based proof assistants to formalize and prove mathematics.

    \item \textbf{In general, do you tend to give your students projects or have them select their own?}  
    
    I give them research topics or directions to explore. For new students, usually, I give them more specific projects to start with so they can get familiar with the research area and techniques. As they are more familiar, they will be able to come up with their own projects. For example, Linhan bounces around several ideas before settling on his work on DNN benchmarking, and Hai and Long work with several students (some not even in our lab) on various projects.
    
   I also aim to determine what my students are interested in and good at, and then help them find projects that fit their interests and strengths. This is not just for research, but also for other things. E.g., in our Thanksgiving party, which I host at my house every year for my lab and their families, I delegate tasks to students based on their strengths and interests: I found out that Linhan is very good at cooking, so I appointed him as our turkey ``czar'' responsible for a 20+ lb turkey roasting for 30 people!

    \item \textbf{Do you have particular projects that you see me working on?}  
    
    It depends on your background and interests, but I do have many ideas to try. However, in recent years, I have been focusing more on DNN reasoning and verification, so something related to that.

    \item \textbf{How much freedom do you think I'd have in selecting my own projects?}  
    
    We need to find projects or research directions that we are both interested in. Once we have agreed on a research direction you will have a lot of freedom to explore and develop your own projects. 

    However, if you want to do something that I am not interested in, then I would rather you find another faculty because I can't guide a topic I don't know or care about.

    \item \textbf{Are there other students you are interested in working with? If so, what would they be working on project-wise?}  
    
    Not sure what ``other students'' means.

    \item \textbf{Would they have their own line of work or contribute to a bigger project/someone else's project?}  
    
    All of my students have their own line of work.  They also collaborate with other lab members and in the beginning, they likely work on projects with other students. But eventually, they will \textit{own} their projects.
\end{enumerate}

\section{Expectations \& Progress}

\begin{enumerate}[label=\arabic*.]
    \item \textbf{What progress does the advisor generally expect from a student in the course of a semester?} [Submission/Publication pace]  
    
    I expect the student to have enough results that they can \textbf{submit} a paper to one of the top conferences each year. It does not have to be accepted because the reviewing could be noisy, but it is crucial to complete the work and submit it. 
    
    You can see how my students publish in the \href{https://roars.dev#publications}{Roars Publications section}. % and also \href{https://github.com/dynaroars/dynaroars.github.io/wiki/Collaborators}{this collaborator page}.

    \item \textbf{What other expectations does the advisor have for their students: time, vacation, paper, project,experiment wise etc.?}

    I want you to \textit{be independent} and \textit{take initiative} in your work. Don't just wait for me to tell you what to do. If you have an idea, try it out yourself and show me the results.
    
    You will have \textit{freedom to come up with solutions and explore ideas in your own way}. Your projects need to be related to my interest, but beyond that, you can do things your way.  
    
    I want you to \textit{ask questions} and \textit{be open to learning new things from others} (e.g., your lab mates, who are probably better resources to learn from than me).  
    
    I want you to listen to me, but also \textit{be independent} and \textit{challenge me}. I am not always right and I want you to tell me when I am wrong. I want you to be able to defend your ideas and work when being challenged by me or others. But you must be able to convince me, through concrete evidence and results, that your idea is better than mine. 
    
    As an example, if I tell you to try \textit{X}, but you think \textit{Y} is better, then you should do both \textit{X} and \textit{Y} and show me both results. It doesn't matter if \textit{X} or \textit{Y} is better, but that you believe in your idea and can get results to support it.

    \item \textbf{When have you given a letter of concern? Why?}  
    
    I haven't given a letter of concern yet, and I hope I never have to.  
    But I'd remind and warn the student if they need to be on track or do not meet the expectation. 
    
    Here is the expected progress for my Ph.D. student. By the end of the second year, you should publish at least a second-author paper. You are doing well if you publish a first-author paper at that time. By the end of the third year, you are considered behind if you do not publish a first-author paper and are seriously behind if no good submissions were made. A typical student in SE/PL will have 4--5 publications, my students often have a lot more than that.
\end{enumerate}

\section{Funding \& Financial Support}

\begin{enumerate}[label=\arabic*.]
    \item \textbf{Where does their funding primarily come from?} [If military/industry-focused funding bothers you, figure this out]  
    
    Most of my funding sources are from NSF and some are from the industry (e.g., gifts from Amazon and Facebook). All of my funded projects are on basic/fundamental scientific research.

    \item \textbf{What are their constraints from their funding source?} [Some restrict research topic and change final deliverables. Some add work — writing progress reports, traveling, preparing presentations for the funding source or engineering overhead for integration]  
    
    Since my funding sources are mainly from NSF or unrestricted from industry, our work has few constraints. Occasionally, I may need your assistance on progress reports and presentations if you are funded through specific grants.

    \item \textbf{If your advisor made you work on a project in their area that you are least interested in (e.g., for a grant) would you still be excited doing work?} [Useful for choosing between advisors]  
    
    It is unlikely that you have to work on something you're not interested in.

    \item \textbf{If you run out of your primary funding for a student how do you expect the student to handle that?} [advisor's responsibility / you'll have to write a grant with me / dept will cover the student / you have to find their own funding]  
    
    One of my responsibilities is finding funding to support my students, and I have been quite lucky to have sufficient funding to support my students (including summer). However, as a fallback, the department can also cover the student with TA-ship. \href{https://github.com/dynaroars/dynaroars.github.io/wiki/About-GMU}{CS@GMU} is \textit{very good} at providing Ph.D. students TA-ships.

    I also encourage my students to apply for fellowships. I will help them with the applications (e.g., give feedback on your statements and provide LORs). These fellowships are not necessary for funding, but they are prestigious and can help your career. I take pride in my students' achievements---sometimes more than in my own. When a student wins a fellowship or an award, it's also a win for me and the entire lab!
\end{enumerate}

\section{Program Requirements}

\begin{enumerate}[label=\arabic*.]
    \item \textbf{What does the quals process look like?}  
    
    The department has a \href{https://cs.gmu.edu/current-students/doctoral-students/comprehensive-exam-new/}{very specific guideline} that students need to follow for the comprehensive exam. In short, it's quite straightforward and the student needs to write a paper describing the research area and problem they want to work on and present that paper.

    \item \textbf{Is there a TA requirement? / How often would I be expected to TA?}

    There is no TA requirement from the university or the department. Nevertheless, I strongly recommend doing a TA at least once or twice during your PhD. TA is a great way to get introduced to teaching, and can really help if you're interested in an academic career. More practically, in some cases not having to support you as a GRA during the normal Spring/Fall semesters would allow me to support you during the Summer.
\end{enumerate}

\section{Recruitment \& Fit / Placement}\label{sec:recruitment}

\begin{enumerate}[label=\arabic*.]
    \item \textbf{Are you taking a student? Do you have funding to take students in this year (or, for which projects)?}  
    
    I am not actively recruiting (my lab already has quite a few students), but I am always looking out for \emph{standout} students. Standing out is difficult to define and subjective---I can only know it when I see it (e.g., see the Reddit example below).

    \item \textbf{What factors will affect whether or not you take a student?}  
    
    Here's a concrete example: One day the CS faculty were talking about a GMU Reddit post from a student who built a web app to identify ``easy A'' classes at GMU (because they find RateMyProfessors not useful). Some profs. found it problematic because this app allows students to find and take easy classes (and they also complained that it doesn't represent their own classes fairly). But I was intrigued.  
    
    I didn't care much about the app itself, but the initiative and technical skill caught my attention: the student finds a problem, comes up with a solution using an unconventional way (from publicly available FOIA data that no one uses), and implements an app that attracts many students (and upset professors!).
    
    That same evening, I mentioned the app on our lab Discord server, and we determined the developer's identity by analyzing the code and deployment details. I emailed the individual right after, we talked the next day, he was very interested in our research, and so I hired him on the spot.


    \item{Do you think our research interests are a good match?}

    You tell me! Take a look at \href{https://roars.dev}{my research lab}.

    \item I'm interested in working with you. Do you think I'd have a good chance of working with you if I come to your university?

    If you come to GMU and want to work with me, let's chat. However, the \emph{best} way to for me to know you (and vice versa) is taking my classes and impress me. I recruit several of my students from my classes (e.g., Linhan, Guolong, and Didier) that way.


\end{enumerate}

\section{Co-Advising \& External Collaborations}

\begin{enumerate}[label=\arabic*.]
    \item \textbf{Would the advisor be interested in co-advising?}  
    
    I was co-advised during my PhD and it worked out, but I \textbf{prefer to be your main or sole advisor}. I prefer that my student have just me as their main advisor, and then have other faculty members as collaborators (not co-advisor). This way, you can get more attention from me while still being mentored by other experts. For example, Hai is advised by me but also collaborates (and therefore mentored) by Matt Dwyer at UVA through our NeuralSAT project. It's also easier for me because I do not want to deal with conflicts between co-advisors, e.g., if they want you to do something that I don't want you to do (and vice versa).

    In addition, I am not into broad \emph{interdisciplinary research}, which would be beneficial for co-advising. I prefer to work on problems in my fields of SE/PL/FM and related ones (e.g., logics, mathematical proofs). So if you want to do interdisciplinary research in areas outside my core expertise (e.g., CS + Bio), then I may not be the best fit for you. While interdisciplinary work has many values (and I admire people who do it well), I have determined that it is not my strength and I prefer to focus on my core areas.

    If you'd like to do something that I am not too interested then I would not want to be your main advisor. I would rather you find another faculty member who is more interested in your work and can help you with it. I think this is a better fit for both of us.

    \item \textbf{How do you handle it when a student outgrows your expertise — they know more than you?}  
    
    I \textit{expect} this to happen. At some point, you will know more than me in your topic of interest. Depending on your projects, I may introduce students to researchers or faculty who are experts in that area and collaborate with them so they can help you in your project. For example, KimHao works with Quoc Sang at Facebook on build systems, and Hai works with Matt Dwyer at UVA on DNN verification. In short, if you become better than me in your area, then I've done my job!  
    
    Note that if you want to change or try a research direction that I don't have much expertise or interest in, then I can't help you much there. I can try to find another faculty member who can help you, but if you want to work on something completely different, then we may need to discuss switching advisors. This is common in academia (almost happened to me during my PhD) and I have no issue with you doing that.
\end{enumerate}

\section{Lab Resources \& Working Environment}

\begin{enumerate}[label=\arabic*.]
    \item \textbf{What equipment are provided to students?}  
    
    I purchase computers and electronics for my students (e.g., computers, laptops, monitors, hard drives, memory, headsets, keyboards, etc). Our lab has several \href{https://github.com/dynaroars/dynaroars.github.io/wiki/Servers}{powerful servers} such as Nvidia Spark DGX (a gift from NVidia), that are shared among the students.  
    Let me put it this way: our lab machines are much better than most other labs and our experiments get done much faster than others.

    \item \textbf{Do students work together in common space? Do students often get meals together?}  
    
    Sort of. Due to office space limitations, my students are sharing space with other SWE PhD students. There are advantages of that as you get to interact with more people outside our lab.

    \item \textbf{Do students often work late?} [Often / only before conference deadlines.]  
    
    We work late mostly before the conference deadline. I also stay up quite late to chat with students if they need. Though it really does not matter to me when you work as long as you are productive.

    \item \textbf{Are there snacks in the lab?}  
    
    Not sure, sometimes I see some snacks. When I was a grad student, I often stole pizza and food from other labs (I have plenty of stories to tell). I often take my groups out to eat (e.g., when working on papers, or end-of-semester gatherings, in fact, I am writing this while preparing for traditional lab Thanksgiving dinner).  
    
    More generally, we work hard but also have \href{https://photos.google.com/share/AF1QipNMHIvnSmq5bjer6I6r1Ddb4yiBt3jgq4yrA7q5lSmo4ePHfmiXhXqhk8IhKf47lA?key=SkJhMjZtYmoyT1Q2aW1ERGpRa2VTclZrbzJVLUpR}{lots of fun}, many of which involve eating and gathering.
\end{enumerate}

\section{Conferences, Travel \& Internships}

\begin{enumerate}[label=\arabic*.]
    \item \textbf{How often do grad students get to attend conferences?} [Pace + What constraints]  
    
    If you have a \textbf{full} paper at a conference, even not as first author, I will support your travel to the conference to present it. I also encourage you and other lab members to go to local conferences or workshops (e.g., PL meetings at schools in the Northeast region such as Stevens, Princeton, UPenn). I often send my students to the annual, week-long Formal Techniques Summer School in San Francisco (CA).  
    
    For full disclosure, I am not a big fan of attending conferences. I think they are useful but I am just not into traveling. I rather stay at home and play with my kids and work on my research. But this is just me and I wholeheartedly support my students to attend conferences, and I introduce them to other researchers through emails and collaboration.

    \item \textbf{How many conferences are students expected to target a year?} [remember pubs $\neq$ submissions]  
    
    On the second year, I expect my students to \textit{submit} \textit{one} first-author paper to a \textit{major} \href{https://roars.dev/csconfs}{conference} each year (e.g., ICSE, FSE, ASE, ISSTA, PLDI, OOPSLA, CAV). 
    On the third year onwards, I expect you to \textit{submit} at least \textit{two} papers each year.

    Essentially, I expect you to get some results to present after a full year of research, and continue and improve that pace throughout your PhD.

    My students often publish more than what was mentioned above, e.g., KimHao published 9 papers as an undergrad, Guolong published 8 papers during his PhD, and Hai already has 5 papers by his 3rd year, etc. 
    
    I strongly believe that students do not know what they are capable of until they push themselves, and they should not remain in their comfort zone (not the level of craziness of \href{https://davidgoggins.com/}{David Goggins}).

    \item \textbf{Is the professor in an 80/20 with a company? Are students' research projects tied to industry funding?}  
    
    No, but I collaborate with researchers from the industry (e.g., Facebook) and have industrial grants/awards (e.g., Amazon). Some students' projects are to solve industrial problems (and they get opportunities to do internships at cool companies).

    \item \textbf{Will it be acceptable/encouraged to intern at a company during the summer?} [Does this change with seniority?]  
    
    Definitely, in fact I encourage you to do that. E.g., KimHao works at Facebook (twice) during the summer.
\end{enumerate}

\section{Culture, Time Off \& Social Life}

\begin{enumerate}[label=\arabic*.]
    \item \textbf{How often do students take time off? Are there lab/department outings/events?}  
    
    \textbf{Time off:} Academia is flexible and I am flexible with my students (as long as they get their work done).
    However I expect my students to be in the lab at least 3 days a week (part of school requirement otherwise they will lose their space, which I think is fair).  But if they need to take time off (e.g., health appointments, personal/family events), they can always tell me and take some time off.  
    
    \textbf{Lab dinner/gathering:} Usually twice a semester (e.g., \href{https://photos.google.com/share/AF1QipNMHIvnSmq5bjer6I6r1Ddb4yiBt3jgq4yrA7q5lSmo4ePHfmiXhXqhk8IhKf47lA?key=SkJhMjZtYmoyT1Q2aW1ERGpRa2VTclZrbzJVLUpR}{group gathering} at the beginning and the end of a semester and during Thanksgiving). Once a year we also hang out with other labs at Round One (bowling and arcade).  
    
    \textbf{Department outings/events:} There are various events throughout the semester in both the SWE group and the department. I highly encourage my students to attend these to have fun and know other people.
\end{enumerate}

\section{Support, Conflicts \& Student Well-Being}

\begin{enumerate}[label=\arabic*.]
    \item \textbf{If one of your students is being treated unfairly — by a collaborator, committee member, or even within the department — how do you handle it?}  
    
    I am very protective of my students. I can be hard on them when they do research with me, but I will stand up for them. Not only obvious cases like unfair treatment but also when I feel they need support, e.g., I would shield them from being overworked or being taken advantage of by others. My priority is the well-being of my students, \textit{period}.

    \item \textbf{How do you deal with students who are struggling? Do you ``let go'' of students who are not making progress?}  
    
    I have never let go of a PhD student yet, and I hope I never have to. Once I take in a PhD student, I see it as my responsibility to get them through the program and into a good career (job that they like).  
    
    Every PhD student will struggle at some point, but I believe my students are capable of pushing through challenges, and I will work with them to make sure that happens. So no, I don't give up on students easily. It's usually easier for them to give up than it is for me to give up on them.
\end{enumerate}

\section{Similar Resources}

\begin{itemize}
    \item \href{https://jbhuang0604.github.io/advisor_guide.html}{Jia-Bin Huang's answers}
    \item \href{https://ideal.umd.edu/blog/Prospective-Students-FAQ}{IDEAL Lab's answers}
\end{itemize}

\end{document}